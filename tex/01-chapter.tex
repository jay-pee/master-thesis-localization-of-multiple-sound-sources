\chapter{Introduction}
\label{chap:introduction}
% With the rise of the virtual assistant like Amazon Alexa, Apple Siri or Google Assistant the trend for voice enabled \ac{IOT} devices and smart speaker began.
% Audiences and consumers are rapidly adopting smart speakers. In the moment (early 2018) the ownership is at 16\% of the U.S. population after 3 years \cite{smartreport}. Gartner, a provider for market research results and analysis, predicts that 75\% of U.S. households will have smart speakers by 2020 \cite{gartner}. Smart speaker are using speech recognition as the \ac{HMI}. Therefore new challenges for speech enhancement algorithm arises to enable a robust speech recognition in difficult acoustic situations. For example the desired speaker is several meters away from the device while other interfering sound sources are present as well. In state of the art speech enhancement systems microphone arrays are deployed to 'conduct' a spatial filtering known as beamforming.
% This is done to capture the desired acoustic signal and suppress other interfering sound sources.
% For applying beamforming it is crucial to know where sound sources are located. Otherwise spatial filtering would corrupt the incoming signal massively.
% Many basic acoustic source localization methods are already developed which however can only identify the most dominant sound source at a time, like \ac{GCC} or \ac{SRP} \cite[Chapter~8]{brandstein2013microphone}.
% Also when using multi-source localization techniques which are computation expensive like \ac{MUSIC} or \ac{CSSM} a post-processing is desirable to estimate the number of sources, classify the raw localization results that represents the speaker and track them over time \cite{krolik1989multiple,1164667}. \\
% The goal of this work is to develop an computational efficient localization algorithm for multiple speaker by utilizing an \ac{UCA+C}. As a baseline, the techniques to be developed should build upon the \ac{SRP} method that can only identify the most dominant sound source at a time. The focus will lie on the development of a postprocessor that classifies the raw localization results to obtain information about the number of sources and the spatial location in the azimuth plane. As extension the baseline the algorithm shall be modified to work with more sophisticated localization methods, that uses spectral localization. The two beamformer \ac{DS} and Capon, shall be considered in the \ac{SRP} method and the whole scene analyses shall be compared to a state of the art method. The algorithm shall be computational efficient and capable to process in real-time and online. The approach shall be modular and extendable to work with multiple localization algorithm. \\
% To delimit the brought topic of localization, in this work only a \ac{UCA+C} with a radius $\SI{42}{mm}$ is considered and the localization shall only be made in the azimuth plane.\\
% The methodological approach is to first choose a classification algorithm that works on batch data and then adapt it to online processing. The algorithm is implemented in the programming language \textit{MATLAB}. After the basic algorithm is developed extensions shall be made to make the algorithm more robust and also work with spectral localization results.\\
% In chapter 2 the basics are stated for the localization and the post-processing. Here, the signal model is first defined an then beamforming is introduced. Based on this the acoustic source localization method \ac{SRP} is discussed. Next, the source classification with a \ac{GMM} and an \ac{EM}-algorithm will be introduced. To adapt the classification method for circular arrays the wrapped Gaussians are stated. In the end of the basic chapter the state of the art will be discussed and the reference algorithm will be presented in more detail.\\
% In chapter 3 the post-processing algorithm is proposed. First the raw Localization method will be clarified and then the corresponding confidence values will be introduced. Next the update of the GMM parameter and the modification to the \ac{EM}-algorithm will be discussed. Mechanism for addition and deletion of classes are introduced in the next section. Last, adjustments to the basic algorithm are introduced to increase the robustness. Herby a method to utilize the spatial aliasing will be presented.\\
% In chapter 4 the evaluation of the developed and the reference algorithm are shown. First, modification to the comparison post-processing algorithm are stated, so that the algorithm works with circular arrays. Then the evaluation and system setup are discussed. Finally results where stated and discussed and a conclusion is made.\\
% In chapter 5 the whole work is briefly summarized and the prospect are stated.



With the rise of virtual assistants like Amazon Alexa, Apple Siri or the Google Assistant the trend of voice enabled \ac{IOT} devices and smart speakers began. Audiences and consumers are rapidly adopting smart speakers. At the moment (early 2018) the ownership is at 16\% of the U.S. population after 3 years \cite{smartreport}. Gartner, a provider for market research results and analysis, predicts that 75\% of U.S. households will have smart speakers by 2020 \cite{gartner}. Smart speakers are using speech recognition as the \ac{HMI}. Therefore, new challenges for speech enhancement algorithms arise to enable a robust speech recognition in difficult acoustic situations. For example the desired speaker may be several meters away from the device while other interfering sound sources are present as well. In state of the art speech enhancement systems microphone arrays are deployed to 'conduct' a spatial filtering known as beamforming. This is done to capture and enhance the desired acoustic signal and suppress other interfering sound sources.\\

For acoustic beamforming it is crucial to know where sound sources are located. Otherwise spatial filtering might degrade quality of the incoming desired signal. Many basic acoustic source localization methods have been developed which, however, can only identify the most dominant sound source at a time. Well known examples are the \ac{GCC} or the \ac{SRP} method \cite[Chapter~8]{brandstein2013microphone}. Also, more complex algorithms such as \ac{MUSIC} or \ac{CSSM} have been proposed for localizing multiple sources simultaneously \cite{krolik1989multiple,1164667}. In order to control beamformer-steering in noisy environments robust \ac{DOA} estimates are required. In practical scenarios the aforementioned localization methods, however, suffer from noise and reverberation and therefore, their \ac{DOA} estimates cannot be used directly to control a beamformer. Furthermore, multi-source localizers such as \ac{MUSIC} are computationally more complex as compared to the \ac{SRP} method. \\

The goal of this work is therefore to develop a computationally efficient localization algorithm for multiple speakers which can be used with a \ac{UCA+C}. Circular microphone arrays allow for azimuth-independent beamsteering and are therefore being used in many smart speaker applications. As a baseline, the techniques to be developed should build upon the broadband \ac{SRP} method as this has proven robust and represents the current state of the art technique at Nuance. \\

The focus lies on the development of a post-processor for the \ac{SRP} as core localizer which classifies the raw localization results into several classes whereas each class represents an acoustic source. In the smart speaker application the azimuth angle is practically more important than the elevation angle. Therefore, source localization is only considered with respect to the azimuth plane in this work. While being simple and robust, the \ac{SRP} method, however, can only identify the most dominant sound source at a time. Even if two sources differ with respect to their spectral content they cannot be distinguished. Therefore, it is desirable to develop a classifier that is not limited to the \ac{SRP} as a core localizer and that can be extended to work with more sophisticated localization methods using spectral localization. \\

The classifier proposed in this thesis uses a \ac{GMM} to represent the raw \ac{DOA} results. To allow for tracking of time varying source positions the mean of each Gaussian is estimated adaptively over time. To achieve this \ac{MAP} estimation known from speaker verification with \acp{GMM} has been adopted \cite{279278}. In the application of smart speakers a $\ang{360}$ localization is required which motivates the use of circular arrays. The classification methods known from the literature, however, do not allow for classification of circular data. This problem is solved here by using a so called \ac{WGMM} which is a periodic extension of a standard \ac{GMM}. As mentioned above, the raw DOA estimates usually suffer from noise and reverberation. Because of these effects the observed data are not always reliable. In order to increase the robustness of the classification process a confidence metric has been incorporated into the proposed classifier. Finally, the effect of spatial aliasing has also been considered to achieve a robust performance for the considered microphone array.\\

All considered algorithms have been implemented in \emph{Matlab}. The described multi-source localization is analyzed extensively using a large number of recordings covering a variety of acoustic scenes. The recorded data have been labeled manually to obtain the ground truth and an analysis framework has been implemented to judge the estimated source positions using a set of metrics. As a reference a well known multi-source localizer as proposed by Madhu has been chosen \cite{madhu2008scalable}. This method also uses a \ac{GMM} but is computationally more complex and does not use \ac{MAP} adaptation. The proposed method is analyzed using 4 different localizers. \\

The work is structured as follows: In chapter 2 the basics are stated for the localization and the post-processing. Here, the signal model is defined and beamforming is introduced. Based on this, the acoustic source localization method \ac{SRP} is discussed. Next, the source classification with a \ac{GMM} and an \ac{EM}-algorithm is introduced. To adapt the classification method for circular arrays the wrapped Gaussians are stated. In the end of the basic chapter the state of the art method is discussed. In chapter 3 the proposed classification algorithm is described in detail. Chapter 4 then contains the evaluation and discusses all results and findings for each considered scenario. Finally, in chapter 5 the whole work is summarized and the prospects are stated.
\endinput
